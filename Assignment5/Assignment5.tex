\documentclass[10pt,a4paper]{article}
\usepackage[utf8]{inputenc}
\usepackage{amsmath}
\usepackage{amsfonts}
\usepackage{amssymb}
\usepackage{listings}
\usepackage{graphicx}
\usepackage{float}
\usepackage{caption}
\usepackage[margin=1in]{geometry}
\lstset{showstringspaces=false,
		breaklines=true,
		postbreak=\raisebox{0ex}[0ex][0ex]{\ensuremath{\hookrightarrow\space}}}			


\renewcommand{\thesubsubsection}{\thesubsection.\alph{subsubsection}}

\begin{document}
\title{Intelligent Systems Assignment 5}
\author{Wessel Becker (1982362) \& Sander ten Hoor (2318555)}
\maketitle

\newcommand{\simplesubfigure}[3]{
  \noindent\begin{minipage}{.5\linewidth}
    \centering
    \includegraphics[width=.5\linewidth]{#1}
    \captionof{figure}{#2}
    \label{#3}
  \end{minipage}%
}
\newcommand{\simplefigure}[3]{
	\begin{figure}[H]
  	\centering
    	\makebox[\textwidth]
    	{
    		\includegraphics[height=0.4\textheight]{#1}
 		} \\
  		\caption{#2}
  		\label{#3}
	\end{figure}
}
\newcommand{\mcode}[1]{
	\lstinputlisting[language=Matlab]{#1}
}

\section{K-means}
\section{Decision tree}
\section{K-nearest neighbours}
\subsection{K-nearest neighbours implementation}
The implementation is available in \ref{ap:knn}

\subsection{Classification for different values of K} \label{ss:class}
\ref{ap:knn_img} contains the resulting images. When K = 1, the two groups of points, the green circles and the red crosses, have their own well-defined territory. As values of K increase, more members of a class seem to be in an area which would be classified as the other. This makes sense, as more matches are necessary to come to a satisfactory conclusion. In an area mostly populated by green circles, a new point taking the five closest distances is very likely to end up with more green circles as nearest points than red crosses.

\subsection{Contending Classes}
With the increased number of classes (2 to 4), the effect described in \ref{ss:class} persists, as would be expected.

In case of multiple contending classes, a choice needs to be made. There are four simple options for this:
\begin{itemize}
\item Lowest/Highest class number
\item A random class
\item Class with the closest point
\item Class with the lowest average distance
\end{itemize}

The first option is rather arbitrary and would prioritize specific classes with no other reason than their number. Choosing a random class gets rid of the arbitrary prioritization, but offers no real basis as to why that class was chosen.
Selecting the class with the closest point gives an argument to support the selection of that specific class, but that distance could be a possible outlier, since it is just one. To take everything into account, selecting the class with the lowest average distance gives the most reliable result.

\appendix
\section{figures}
\subsection{K-means}
  \simplesubfigure{./matlab/img/x2}{2 clusters K=2}{fig:x2}%
  \simplesubfigure{./matlab/img/x4}{2 clusters K=4}{fig:x4}%
  \simplesubfigure{./matlab/img/x8}{2 clusters K=8}{fig:x8}%
  \simplesubfigure{./matlab/img/y2}{1 long cluster K=2}{fig:y2}%
  \simplesubfigure{./matlab/img/y4}{1 long cluster K=4}{fig:y4}%
  \simplesubfigure{./matlab/img/y8}{1 long cluster K=8}{fig:y8}%
  \simplesubfigure{./matlab/img/z2}{uniform data K=2}{fig:z12}%
  \simplesubfigure{./matlab/img/z22}{uniform data K=2}{fig:z22}%
  \simplesubfigure{./matlab/img/z4}{uniform data K=4}{fig:z4}%
  \simplesubfigure{./matlab/img/z8}{uniform data K=8}{fig:z18}%
  \simplesubfigure{./matlab/img/z28}{uniform data K=8}{fig:z28}%
\subsection{Decision tree}
\simplefigure{./tree}{decision tree generated with the c4 algorithm using weka}{fig:tree}

\subsection{K-nearest neighbours}\label{ap:knn_img}
\subsubsection{Number of classes = 2}
\simplefigure{./matlab/img/3_1_2.png}{K = 1, Nr of classes = 2}{fig:3_1_2}
\simplefigure{./matlab/img/3_3_2.png}{K = 3, Nr of classes = 2}{fig:3_3_2}
\simplefigure{./matlab/img/3_5_2.png}{K = 5, Nr of classes = 2}{fig:3_5_2}
\simplefigure{./matlab/img/3_7_2.png}{K = 7, Nr of classes = 2}{fig:3_7_2}

\subsubsection{Number of classes = 4}
\simplefigure{./matlab/img/3_1_4.png}{K = 1, Nr of classes = 4}{fig:3_1_4}
\simplefigure{./matlab/img/3_3_4.png}{K = 3, Nr of classes = 4}{fig:3_3_4}
\simplefigure{./matlab/img/3_5_4.png}{K = 5, Nr of classes = 4}{fig:3_5_4}
\simplefigure{./matlab/img/3_7_4.png}{K = 7, Nr of classes = 4}{fig:3_7_4}

\section{code}
\subsection{k-means}
\mcode{./matlab/simpleKMeans.m}
\mcode{./matlab/plotKMeans.m}
\subsection{K-nearest neighbours}\label{ap:knn}
\mcode{./matlab/KNN.m}
\mcode{./matlab/simpleFrequencies.m}

\end{document}