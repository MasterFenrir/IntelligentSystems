\documentclass[10pt,a4paper]{article}
\usepackage[utf8]{inputenc}
\usepackage{amsmath}
\usepackage{amsfonts}
\usepackage{amssymb}
\usepackage{listings}
\usepackage{graphicx}
\usepackage{caption}
\usepackage{subcaption}
\usepackage{float}
\lstset{showstringspaces=false,
		breaklines=true,
		postbreak=\raisebox{0ex}[0ex][0ex]{\ensuremath{\hookrightarrow\space}}}
    	
\begin{document}
\title{Intelligent Systems Assignment 3}
\author{Wessel Becker (1982362) \& Sander ten Hoor (2318555)}
\maketitle

\section{Matlab Code}
The following code was created for edge detection and boundary detection.

\subsection{simpleDifferentiation.m}
\lstinputlisting[language=Matlab]{./edgedetection/simpleDifferentiation.m}

\subsection{robert.m}
\lstinputlisting[language=Matlab]{./edgedetection/robert.m}

\subsection{sobel.m}
\lstinputlisting[language=Matlab]{./edgedetection/sobel.m}

\subsection{prewitt.m}
\lstinputlisting[language=Matlab]{./edgedetection/prewitt.m}

%\subsection{cannyEdgedetection.m}
%\lstinputlisting[language=Matlab]{./edgedetection/cannyEdgedetection.m}

\subsection{boundaryExtraction.m}
\lstinputlisting[language=Matlab]{./edgedetection/boundaryExtraction.m}\label{list:boundaryExtraction}

\section{Edge Detection}
While all the results suffer from noise, the Simple Differentiation and the Roberts algorithm become nearly unrecognisable, while the others still show clear lines.


\section{Boundary Extraction}
For exercise 2.2 a simple boundary extractor was build as can be found in the code listings \ref{list:boundaryExtraction}. This very simple boundary extractor can be run with a variable threshold for hit detection. And works by first eroding anything that could be the inside of a shape by doing an erosion with a 3 * 3 structuring element with the origo in the center. This will erode any pixel that is not surrounded by other accepted pixels but is accepted, or in other words the boundaries. If we then subtract this image from our original binary image of accepted values we are left with the pixels we eroded before or in other words the boundaries as seen in \ref{fig:}.

\section{Canny Edge Detection}
Exercise 2.3 asked us to analyse the impact of noise, more specifically Gaussian noise, on the Canny Edge Detection algorithm. We also consider the influence of some of the tuning variables for the algorithm, most importantly the sigma variable, further referred to as s, which is the standard deviation for the Gaussian blur filter. We will also take a  look at the influence of changing the threshold values for the acceptance of weak and strong edges. This variable will be referred to as t and it is represented as a factor of the weak edge threshold and the strong edge threshold respectively.

It is expected that when the Gaussian noise is increased the edge detection algorithm will start to pick more false edges. It is also expected that increasing s and therefore also the size of the Gaussian filter will reduce the amount of false edges when the level of noise is increased, however probably at some cost to the amount of true edges being detected.

Further expected is, that choosing a lower value for the weak edge threshold will increase the amount of weaker edges connected to strong edges that is detected, but increasing the amount of false positives as it gets lower as well, especially in an image with more Gaussian noise. 

And lastly we expect that the lower the value chosen for the strong edge threshold the more edges we will find in general, however making this value to low should lead to many false positives for the edges especially in images with allot of noise.

\section{Surround Suppression}
Surround suppression is a way of enhancing contours and region boundaries in images with texture. This can be used in combination with other edge detection algorithms such as Canny Edge Detection. 

There are two types of edge detection, namely isotropic and anisotropic. Isotropic surround suppression will suppress the gradient magnitude for pixels which already have many edge pixels nearby, weighting closer pixels more heavily then far away pixels. this will suppress edges inside of textures but not the textures boundaries even if the textured area has a chaotic pattern like sand, there will be a strong suppression. Isotropic Surround Suppression will, however not be able to strengthen edges between differently oriented textures.

Anisotropic Surround Suppression will do something similar to Isotropic Surround Suppression, except that it will actually take into account the orientation of the gradient in the surroundings. Which makes it possible to find transitions between differently oriented textures, however the isotropic Surround Suppression will not mark the transition form a textured part of the image to a non textured part as strongly.

We can clearly see some of these differences looking at the Kanisza and popout images after these forms of suppression have been applied to them. As an example we can see that the isotropic filtering will suppress edges that near many edges, while clearly leaving the edges of the area containing many edge, see \ref{fig:kanizsa_L2_a4_k11_k24}. When we look at the same image with anisotropic surround suppression \ref{fig:kanizsa_ANISO_a4_k11_k24} we see that the edges of this area are not as clearly defined but instead any area which has differently oriented edges has become very strongly expressed. Similar effect can be seen in the popout images, see \ref{fig:popout_LINF_a3_k11_k24} en onward, however not as clearly as with the Kanizsa image. It would have probably been more interesting if we had chosen to do the suppression on this image over a larger area by increasing the K1 and K2 variables. Which could have caused the anisotropic Surround Suppression to clearly indicate the edges of the center bar which has a different orientation.

We can also clearly see the effect of the alpha variable on the strength of the suppression by examining for example the alpha = 2 and alpha = 5 images for isotropic Surround suppression, see \ref{fig:kanizsa_L1_a2_k11_k24} and \ref{fig:kanizsa_L1_a5_k11_k24}.



\section{Work done}

\section{Edge Detection images}
\subsection{Simple Differentiation}

\begin{figure}
  \centering
    \makebox[\textwidth]{\includegraphics{./edgedetection/images/sd_no_noise}} \\
  \caption{Simple Differentiation, no noise}
  \label{fig:sd_no_noise}
\end{figure}

\begin{figure}
  \centering
    \makebox[\textwidth]{\includegraphics{./edgedetection/images/sd_001_noise}} \\
  \caption{Simple Differentiation, Gaussian noise with variance = 0.001}
  \label{fig:sd_001}
\end{figure}

\begin{figure}
  \centering
    \makebox[\textwidth]{\includegraphics{./edgedetection/images/sd_005_noise}} \\
  \caption{Simple Differentiation, Gaussian noise with variance = 0.005}
  \label{fig:sd_005}
\end{figure}

\subsection{Roberts}
\begin{figure}[H]
  \centering
  
  \begin{subfigure}{.5\textwidth}
    \centering
    \includegraphics[width=.9\textwidth]{./edgedetection/images/robert_no_noise}
    \caption{Roberts, no noise}
    \label{fig:robert_no_noise}
  \end{subfigure}%
  \begin{subfigure}{.5\textwidth}
    \centering
    \includegraphics[width=.9\textwidth]{./edgedetection/images/robert_001_noise}
    \caption{Roberts, Gaussian noise with variance = 0.001}
    \label{fig:robert_001}
  \end{subfigure}\\%
  \begin{subfigure}{.5\textwidth}
    \centering
    \includegraphics[width=.9\textwidth]{./edgedetection/images/robert_005_noise}
    \caption{Roberts, Gaussian noise with variance = 0.005}
    \label{fig:robert_005}
  \end{subfigure}%
  
\end{figure}


\subsection{Sobel}
\begin{figure}[H]
  \centering
  
  \begin{subfigure}{.5\textwidth}
    \centering
    \includegraphics[width=.9\textwidth]{./edgedetection/images/sobel_no_noise}
    \caption{Sobel, no noise}
    \label{fig:sobel_no_noise}
  \end{subfigure}%
  \begin{subfigure}{.5\textwidth}
    \centering
    \includegraphics[width=.9\textwidth]{./edgedetection/images/sobel_001_noise}
    \caption{Sobel, Gaussian noise with variance = 0.001}
    \label{fig:sobel_001}
  \end{subfigure}\\%
  \begin{subfigure}{.5\textwidth}
    \centering
    \includegraphics[width=.9\textwidth]{./edgedetection/images/sobel_005_noise}
    \caption{Sobel, Gaussian noise with variance = 0.005}
    \label{fig:sobel_005}
  \end{subfigure}%
  
\end{figure}


\subsection{Prewitt}

\begin{figure}
  \centering
    \makebox[\textwidth]{\includegraphics{./edgedetection/images/prewitt_no_noise}} \\
  \caption{Prewitt, no noise}
  \label{fig:prewitt_no_noise}
\end{figure}

\begin{figure}
  \centering
    \makebox[\textwidth]{\includegraphics{./edgedetection/images/prewitt_001_noise}} \\
  \caption{Prewitt, Gaussian noise with variance = 0.001}
  \label{fig:prewitt_001}
\end{figure}

\begin{figure}
  \centering
    \makebox[\textwidth]{\includegraphics{./edgedetection/images/prewitt_005_noise}} \\
  \caption{Prewitt, Gaussian noise with variance = 0.005}
  \label{fig:prewitt_005}
\end{figure}

\section{Boundary detection}
\begin{figure}[H]
  \centering
    \makebox[\textwidth]{\includegraphics[width=.9\textwidth]{./edgedetection/boundary_extraction/boundary_extraction}} \\
  \caption{Scatter, n = 0.1, 2 prototypes}
  \label{fig:n01_k2}
\end{figure}

\section{Canny images}
\subsection{No noise}
\begin{figure}[H]
  \centering
  
  \begin{subfigure}{.5\textwidth}
    \centering
    \includegraphics[width=.9\textwidth]{./edgedetection/canny_no_noise/no_noise_t_00625_01562}
    \caption{No noise, t = [0.0625, 0.1562], s = sqrt(2)}
    \label{fig:cany_no_noise_default}
  \end{subfigure}%
  \begin{subfigure}{.5\textwidth}
    \centering
    \includegraphics[width=.9\textwidth]{./edgedetection/canny_no_noise/no_noise_s_srt_1}
    \caption{No noise, t = [0.0625, 0.1562], s = sqrt(1)}
    \label{fig:cany_no_noise_low_s}
  \end{subfigure}\\%
  \begin{subfigure}{.5\textwidth}
    \centering
    \includegraphics[width=.9\textwidth]{./edgedetection/canny_no_noise/no_noise_s_srt_3}
    \caption{No noise, t = [0.0625, 0.1562], s = sqrt(3)}
    \label{fig:cany_no_noise_high_s}
  \end{subfigure}%
  \begin{subfigure}{.5\textwidth}
    \centering
    \includegraphics[width=.9\textwidth]{./edgedetection/canny_no_noise/no_noise_t_003125_01562}
    \caption{No noise, t = [0.03125, 0.1562], s = sqrt(2)}
    \label{fig:cany_no_noise_low_low_t}
  \end{subfigure}%
  
\end{figure}

\begin{figure}[H]
  \centering
  
  \begin{subfigure}{.5\textwidth}
    \centering
    \includegraphics[width=.9\textwidth]{./edgedetection/canny_no_noise/no_noise_t_009375_01562}
    \caption{No noise, t = [0.09375, 0.1562], s = sqrt(2)}
    \label{fig:cany_no_noise_high_low_t}
  \end{subfigure}%
  \begin{subfigure}{.5\textwidth}
    \centering
    \includegraphics[width=.9\textwidth]{./edgedetection/canny_no_noise/no_noise_t_00625_00781}
    \caption{No noise, t = [0.0625, 0.0781], s = sqrt(2)}
    \label{fig:cany_no_noise_low_high_t}
  \end{subfigure}\\%
  \begin{subfigure}{.5\textwidth}
    \centering
    \includegraphics[width=.9\textwidth]{./edgedetection/canny_no_noise/no_noise_t_00625_02343}
    \caption{No noise, t = [0.0625, 0.2343], s = sqrt(2)}
    \label{fig:cany_no_noise_high_high_t}
  \end{subfigure}%
  
\end{figure}




\subsection{Medium noise}
\begin{figure}[H]
  \centering
  
  \begin{subfigure}{.5\textwidth}
    \centering
    \includegraphics[width=.9\textwidth]{./edgedetection/medium_noise/m_noise_def}
    \caption{Medium noise, default}
    \label{fig:m_noise_def}
  \end{subfigure}%
  
  \begin{subfigure}{.5\textwidth}
    \centering
    \includegraphics[width=.9\textwidth]{./edgedetection/medium_noise/m_noise_sens_l_thres}
    \caption{Medium noise, sensitive, low threshold}
    \label{fig:m_noise_sens_l_thres}
  \end{subfigure}\\%
  
  \begin{subfigure}{.5\textwidth}
    \centering
    \includegraphics[width=.9\textwidth]{./edgedetection/medium_noise/m_noise_sens_h_thres}
    \caption{Medium noise, sensitive, high threshold}
    \label{fig:m_noise_sens_h_thres}
  \end{subfigure}%
  
  \begin{subfigure}{.5\textwidth}
    \centering
    \includegraphics[width=.9\textwidth]{./edgedetection/medium_noise/m_noise_insens_l_thres}
    \caption{Medium noise, insensitive, low threshold}
    \label{fig:m_noise_insens_l_thres}
  \end{subfigure}\\%
  
  \begin{subfigure}{.5\textwidth}
    \centering
    \includegraphics[width=.9\textwidth]{./edgedetection/medium_noise/m_noise_insens_h_thres}
    \caption{Medium noise, insensitive, high threshold}
    \label{fig:m_noise_insens_h_thres}
  \end{subfigure}%
  
    \begin{subfigure}{.5\textwidth}
    \centering
    \includegraphics[width=.9\textwidth]{./edgedetection/medium_noise/m_noise_insens_l_thres}
    \caption{Medium noise, low sigma, sqrt(1)}
    \label{fig:m_noise_insens_l_thres}
  \end{subfigure}\\%
  
  \begin{subfigure}{.5\textwidth}
    \centering
    \includegraphics[width=.9\textwidth]{./edgedetection/medium_noise/m_noise_h_sigma_sqr3}
    \caption{Medium noise, high sigma, sqrt(3)}
    \label{fig:m_noise_h_sigma_sqr3}
  \end{subfigure}%
  
\end{figure}


\subsection{Heavy noise}
\begin{figure}[H]
  \centering
  
  \begin{subfigure}{.5\textwidth}
    \centering
    \includegraphics[width=.9\textwidth]{./edgedetection/heavy_noise/h_noise_def}
    \caption{Heavy noise, default}
    \label{fig:h_noise_def}
  \end{subfigure}%
  
  \begin{subfigure}{.5\textwidth}
    \centering
    \includegraphics[width=.9\textwidth]{./edgedetection/heavy_noise/h_noise_sens_l_thres}
    \caption{Heavy noise, sensitive, low threshold}
    \label{fig:h_noise_sens_l_thres}
  \end{subfigure}\\%
  
  \begin{subfigure}{.5\textwidth}
    \centering
    \includegraphics[width=.9\textwidth]{./edgedetection/heavy_noise/h_noise_sens_h_thres}
    \caption{Heavy noise, sensitive, high threshold}
    \label{fig:h_noise_sens_h_thres}
  \end{subfigure}%
  
  \begin{subfigure}{.5\textwidth}
    \centering
    \includegraphics[width=.9\textwidth]{./edgedetection/heavy_noise/h_noise_insens_l_thres}
    \caption{Heavy noise, insensitive, low threshold}
    \label{fig:h_noise_insens_l_thres}
  \end{subfigure}\\%
  
  \begin{subfigure}{.5\textwidth}
    \centering
    \includegraphics[width=.9\textwidth]{./edgedetection/heavy_noise/h_noise_insens_h_thres}
    \caption{Heavy noise, insensitive, high threshold}
    \label{fig:h_noise_insens_h_thres}
  \end{subfigure}%
  
    \begin{subfigure}{.5\textwidth}
    \centering
    \includegraphics[width=.9\textwidth]{./edgedetection/heavy_noise/h_noise_insens_l_thres}
    \caption{Heavy noise, low sigma, sqrt(1)}
    \label{fig:h_noise_insens_l_thres}
  \end{subfigure}\\%
  
  \begin{subfigure}{.5\textwidth}
    \centering
    \includegraphics[width=.9\textwidth]{./edgedetection/heavy_noise/h_noise_h_sigma_sqr3}
    \caption{Heavy noise, high sigma, sqrt(3)}
    \label{fig:h_noise_h_sigma_sqr3}
  \end{subfigure}%
  
\end{figure}



\section{Surround suppression images}
\subsection{kanisza}
\begin{figure}[H]
  \centering

\begin{subfigure}{.7\textwidth}
    \centering
    \includegraphics[width=.9\textwidth]{./canny/kanizsa}
    \caption{Original kanizsa image}
    \label{fig:kanizsa}
  \end{subfigure}%  

\end{figure}
\subsubsection{kanisza, alpha = 2}
% Kanizsa, alpha = 2
\begin{figure}[H]
  \centering

\begin{subfigure}{.7\textwidth}
    \centering
    \includegraphics[width=.9\textwidth]{./canny/kanizsa}
    \caption{Original kanizsa image}
    \label{fig:kanizsa}
  \end{subfigure}%  
  
    \begin{subfigure}{.7\textwidth}
    \centering
    \includegraphics[width=.9\textwidth]{./canny/kanizsa_LINF_a2_k11_k24}
    \caption{kanizsa, Isotropic L-infinity norm. $\alpha$ = 2, K1 = 1, K2 = 4}
    \label{fig:kanizsa_LINF_a2_k11_k24}
  \end{subfigure}%

\end{figure}

\begin{figure}[H]
\centering

  \begin{subfigure}{.7\textwidth}
    \centering
    \includegraphics[width=.9\textwidth]{./canny/kanizsa_L1_a2_k11_k24}
    \caption{kanizsa, Isotropic L1 norm. $\alpha$ = 2, K1 = 1, K2 = 4}
    \label{fig:kanizsa_L1_a2_k11_k24}
  \end{subfigure}%
  
  \begin{subfigure}{.7\textwidth}
    \centering
    \includegraphics[width=.9\textwidth]{./canny/kanizsa_L2_a2_k11_k24}
    \caption{kanizsa, Isotropic L2 norm. $\alpha$ = 2, K1 = 1, K2 = 4}
    \label{fig:kanizsa_L2_a2_k11_k24}
  \end{subfigure}\\%
 \end{figure}

\begin{figure}[H]
\centering 
  \begin{subfigure}{.7\textwidth}
    \centering
    \includegraphics[width=.9\textwidth]{./canny/kanizsa_ANISO_a2_k11_k24}
    \caption{kanizsa, Anisotropic. $\alpha$ = 2, K1 = 1, K2 = 4}
    \label{fig:kanizsa_ANISO_a2_k11_k24}
  \end{subfigure}%
  
\end{figure}
\subsubsection{kanisza, alpha = 4}
% Kanizsa, alpha = 2
\begin{figure}[H]
  \centering
  
    \begin{subfigure}{.7\textwidth}
    \centering
    \includegraphics[width=.9\textwidth]{./canny/kanizsa_LINF_a4_k11_k24}
    \caption{kanizsa, Isotropic L-infinity norm. $\alpha$ = 4, K1 = 1, K2 = 4}
    \label{fig:kanizsa_LINF_a4_k11_k24}
  \end{subfigure}%

\end{figure}

\begin{figure}[H]
\centering

  \begin{subfigure}{.7\textwidth}
    \centering
    \includegraphics[width=.9\textwidth]{./canny/kanizsa_L1_a4_k11_k24}
    \caption{kanizsa, Isotropic L1 norm. $\alpha$ = 4, K1 = 1, K2 = 4}
    \label{fig:kanizsa_L1_a4_k11_k24}
  \end{subfigure}%
  
  \begin{subfigure}{.7\textwidth}
    \centering
    \includegraphics[width=.9\textwidth]{./canny/kanizsa_L2_a4_k11_k24}
    \caption{kanizsa, Isotropic L2 norm. $\alpha$ = 4, K1 = 1, K2 = 4}
    \label{fig:kanizsa_L2_a4_k11_k24}
  \end{subfigure}\\%
 \end{figure}

\begin{figure}[H]
\centering 
  \begin{subfigure}{.7\textwidth}
    \centering
    \includegraphics[width=.9\textwidth]{./canny/kanizsa_ANISO_a4_k11_k24}
    \caption{kanizsa, Anisotropic. $\alpha$ = 4, K1 = 1, K2 = 4}
    \label{fig:kanizsa_ANISO_a4_k11_k24}
  \end{subfigure}%
  
\end{figure}
\subsubsection{kanisza, alpha = 5}
% Kanisza, alpha = 5
\begin{figure}
  \centering
    \makebox[\textwidth]{\includegraphics{./canny/kanizsa_LINF_a5_k11_k24}} \\
  \caption{kanizsa, Isotropic L-infinity norm. $\alpha$ = 5, K1 = 1, K2 = 4}
  \label{fig:kanizsa_LINF_a5_k11_k24}
\end{figure}

\begin{figure}
  \centering
    \makebox[\textwidth]{\includegraphics{./canny/kanizsa_L1_a5_k11_k24}} \\
  \caption{kanizsa, Isotropic L1 norm. $\alpha$ = 5, K1 = 1, K2 = 4}
  \label{fig:kanizsa_L1_a5_k11_k24}
\end{figure}

\begin{figure}
  \centering
    \makebox[\textwidth]{\includegraphics{./canny/kanizsa_L2_a5_k11_k24}} \\
  \caption{kanizsa, Isotropic L2 norm. $\alpha$ = 5, K1 = 1, K2 = 4}
  \label{fig:kanizsa_L2_a5_k11_k24}
\end{figure}

\begin{figure}
  \centering
    \makebox[\textwidth]{\includegraphics{./canny/kanizsa_ANISO_a5_k11_k24}} \\
  \caption{kanizsa, Anisotropic. $\alpha$ = 5, K1 = 1, K2 = 4}
  \label{fig:kanizsa_L2_a5_k11_k24}
\end{figure}

\subsection{popout}
\begin{figure}[H]
  \centering
	\begin{subfigure}{.7\textwidth}
    \centering
    \includegraphics[width=.9\textwidth]{./canny/popout}
    \caption{Original popout image}
    \label{fig:popout}
  \end{subfigure}%    

\end{figure}
\subsubsection{popout, alpha = 3}
% Popout, alpha = 3
\begin{figure}
  \centering
    \makebox[\textwidth]{\includegraphics{./canny/popout}} \\
  \caption{Original popout image}
  \label{fig:popout}
\end{figure}

\begin{figure}
  \centering
    \makebox[\textwidth]{\includegraphics{./canny/popout_LINF_a3_k11_k24}} \\
  \caption{popout, Isotropic L-infinity norm. $\alpha$ = 3, K1 = 1, K2 = 4}
  \label{fig:popout_LINF_a3_k11_k24}
\end{figure}

\begin{figure}
  \centering
    \makebox[\textwidth]{\includegraphics{./canny/popout_L1_a3_k11_k24}} \\
  \caption{popout, Isotropic L1 norm. $\alpha$ = 3, K1 = 1, K2 = 4}
  \label{fig:popout_L1_a3_k11_k24}
\end{figure}

\begin{figure}
  \centering
    \makebox[\textwidth]{\includegraphics{./canny/popout_L2_a3_k11_k24}} \\
  \caption{popout, Isotropic L2 norm. $\alpha$ = 3, K1 = 1, K2 = 4}
  \label{fig:popout_L2_a3_k11_k24}
\end{figure}

\begin{figure}
  \centering
    \makebox[\textwidth]{\includegraphics{./canny/popout_ANISO_a3_k11_k24}} \\
  \caption{popout, Anisotropic. $\alpha$ = 3, K1 = 1, K2 = 4}
  \label{fig:popout_ANISO_a3_k11_k24}
\end{figure}
\subsubsection{popout, alpha = 4}
% Popout, alpha = 4

\begin{figure}
  \centering
    \makebox[\textwidth]{\includegraphics{./canny/popout_LINF_a4_k11_k24}} \\
  \caption{popout, Isotropic L-infinity norm. $\alpha$ = 3, K1 = 1, K2 = 4}
  \label{fig:popout_LINF_a4_k11_k24}
\end{figure}

\begin{figure}
  \centering
    \makebox[\textwidth]{\includegraphics{./canny/popout_L1_a4_k11_k24}} \\
  \caption{popout, Isotropic L1 norm. $\alpha$ = 3, K1 = 1, K2 = 4}
  \label{fig:popout_L1_a4_k11_k24}
\end{figure}

\begin{figure}
  \centering
    \makebox[\textwidth]{\includegraphics{./canny/popout_L2_a4_k11_k24}} \\
  \caption{popout, Isotropic L2 norm. $\alpha$ = 3, K1 = 1, K2 = 4}
  \label{fig:popout_L2_a4_k11_k24}
\end{figure}

\begin{figure}
  \centering
    \makebox[\textwidth]{\includegraphics{./canny/popout_ANISO_a4_k11_k24}} \\
  \caption{popout, Anisotropic. $\alpha$ = 3, K1 = 1, K2 = 4}
  \label{fig:popout_ANISO_a4_k11_k24}
\end{figure}

\end{document}