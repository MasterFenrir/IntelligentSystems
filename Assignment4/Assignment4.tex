\documentclass[10pt,a4paper]{article}
\usepackage[utf8]{inputenc}
\usepackage{amsmath}
\usepackage{amsfonts}
\usepackage{amssymb}
\usepackage{listings}
\usepackage{graphicx}
\usepackage{caption}
\usepackage{subcaption}
\usepackage{float}
\usepackage[margin=1in]{geometry}
\lstset{showstringspaces=false,
		breaklines=true,
		postbreak=\raisebox{0ex}[0ex][0ex]{\ensuremath{\hookrightarrow\space}}}			


\renewcommand{\thesubsubsection}{\thesubsection.\alph{subsubsection}}

\begin{document}
\title{Intelligent Systems Assignment 4}
\author{Wessel Becker (1982362) \& Sander ten Hoor (2318555)}
\maketitle

\newcommand{\simplefigure}[3]{
	\begin{figure}[H]
  	\centering
    	\makebox[\textwidth]
    	{
    		\includegraphics[width=.6\textwidth]{#1}
 		} \\
  		\caption{#2}
  		\label{#3}
	\end{figure}
}
\newcommand{\mcode}[1]{
	\lstinputlisting[language=Matlab]{#1}
}

\section{1D distrubitions}
In this exercise we examined a dataset of male en female body lengths, the goal being to classify data points into either being of male or of female sex. In order to achieve this in this case we will simply examine the data visually and eyeball a good decision point.

\subsection{Data visualization}
First of we will visualize the data, we do this by drawing a combined graph containing a histogram plot of both heights for women and man. This gives us a clear insight in the way that these heights are distributed as seen figure \ref{fig:1_1}. 

\subsection{Avoiding misclassification on of man}
We then examine the way we could choose to decide whether a new data point is male or female. If we choose the decision criteria as 170 cm with the assumption that everyone larger than 170 cm is male, there would be no men classified incorrectly, for our existing data and if we assume the data to be normally distributed as the histograms seem to indicate, see figure \ref{fig:1_1}. the change of incorrectly assumed males would still stay very low. However, several women would be misclassified. Counting the bars, which suffices in this small sample, representing women right of 170 cm gives us 29 incorrectly classified women, see figure \ref{fig:1_1}, which is allot so unless we find it really important not to miss classify man we can probably do better.

\subsection{minimizing the overall number of miss classifications}
In the case of classification humans by sex we probably want to minimize the overall number of miss classifications on both sides. If the decision criteria would be 178 cm, with everyone larger than 178 cm being male, the lowest number of classification errors would occur according to these histograms. It results in 8 classification errors for women and 4 for men, see figure \ref{fig:1_1}. Making for 12 miss classifications int total much better then the 29 with the previous criteria. It is possible to improve the results for new data points if we consider de way we expect them to be distributed.

\section{2D distributions}
In this exercise a dataset of body heights and hair length's is examined width the goal of finding a classification boundary to classify the dataset into male and female sexes. This is done via visualization.

\subsection{visualization}
In order to get clear visualization of the data which which we are working we make a scatter plot of the data points. In this scatter plot we can clearly see two distinct clusters, see figure \ref{fig:2_1}, which can be assumed to belong to the male and female sex classes.

\subsection{Decision boundary}
In order to determine a decision boundary in this scenario we can also give the fact stated in the exercise that in general men have shorter hair and are taller then women. we also know that we have the same number of male and female data points, neatly coinciding with our two clusters. In order to make as little miss qualifications as possible and with the spread of both cluster being similar. We can then eyeball the line as seen in figure \ref{fig:2_2}. This line was chosen because it divides the two obvious clusters, being as far away from the centres of both clusters as possible. Th line is drawn far away from the cluster centres because this is where the density of data from a certain class is expected to be the highest by keeping the line as far away from these centres we minimize the amount of outlying data points that we miss classify.

\section{}
\subsection{}
\subsection{}
Both fishes, of 8 cm and 20 cm, would be classified as a seabass. However, the error chance would be higher for the fish of 8 cm, since the posterior probabilities are closer to each other than at 20 cm.

\section{}
\stepcounter{subsection}
\subsection{}
See Appendix \ref{a:plotHDHistogram}.
\subsection{}

\section{Work done}
Exercise 1 and 2 were mostly done by Sander, exercise 3 was done together and 4 was mostly done by Wessel, with input from Sander.

\appendix
\section{graphs}
\subsection{Exercise 1 graphs}
\simplefigure{./images/1_1}{Histogram of height, created using \ref{a:plotHistogram}}{fig:1_1}

\subsection{Exercise 2 graphs}
\simplefigure{./images/2_1}{Body length vs hair length, created using \ref{a:plotLengthAgainstHair}}{fig:2_1}
\simplefigure{./images/2_2}{The decision boundary}{fig:2_2}

\subsection{Exercise 3 graphs}
\simplefigure{./images/3_1}{Posterior probablity, created using \ref{a:plotFish}}{fig:3_1}

\subsection{Exercise 4 graphs}
\simplefigure{./images/4_3}{Histograms of the iris HD, created using \ref{a:plotHDHistogram}}{fig:4_3}

\section{Matlab code}
\subsection{plotHistogram.m}\label{a:plotHistogram}
\mcode{./matlab/plotHistogram.m}

\subsection{plotLengthAgainstHair.m}\label{a:plotLengthAgainstHair}
\mcode{./matlab/plotLengthAgainstHair.m}

\subsection{plotFish.m}\label{a:plotFish}
\mcode{./matlab/plotFish.m}

\subsection{plotHDHistogram.m}\label{a:plotHDHistogram}
\mcode{./matlab/plotHDHistogram.m}

\end{document}